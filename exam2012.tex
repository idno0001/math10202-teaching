\documentclass[english,12pt,a4paper]{scrartcl}
\usepackage{multicol}
\usepackage{mdframed}
\usepackage[cm]{anton}

\renewcommand{\vec}[1]{\bm{#1}}

\title{MATH10202 linear algebra}
\subtitle{2012 exam}
\author{}
\date{\vspace{-5ex}}

\DeclareMathOperator{\spn}{span}
\newcommand\spanset[1]{\ensuremath{\spn\set{#1}}}

\begin{document}
\maketitle

\begin{itemize}
  \item[A3 (i)] Let $A$ be the following $4 \times 4$ matrix:
    \[
      A =
      \begin{pmatrix}
        1 & 2 & 2 & 3 \\
        2 & 3 & 3 & 4 \\
        3 & 4 & 4 & 5 \\
        4 & 5 & 5 & 6
      \end{pmatrix}.
    \]
    Find a lower triangular matrix $L$ and an upper triangular matrix $U$ so 
    that $A = LU$.

    \textbf{Rough solution.} We want to write $A$ as the product of a lower and 
    upper triangular matrix. Thus, your first instinct should be to row reduce 
    $A$ to an upper triangular matrix. Don't worry about getting it to 
    (reduced) row-echelon form---here, we don't need leading $1$s in the rows.  

    Begin with
    \[
      \begin{pmatrix}
        1 & 2 & 2 & 3 \\
        2 & 3 & 3 & 4 \\
        3 & 4 & 4 & 5 \\
        4 & 5 & 5 & 6
      \end{pmatrix}
      \refrel{$\substack{r_2 \to r_2 - 2r_1 \\ r_3 \to r_3 - 3r_1 \\ r_4 \to 
      r_4 - 4r_1}$}{\to}
      \begin{pmatrix}
        1 & 2 & 2 & 3 \\
        0 & -1 & -1 & -2 \\
        0 & -2 & -2 & -4 \\
        0 & -3 & -3 & -6
      \end{pmatrix}
      \refrel{$\substack{r_3 \to r_3 - 2r_2 \\ r_4 \to r_4 -3r_2}$}{\to}
      \begin{pmatrix}
        1 & 2 & 2 & 3 \\
        0 & -1 & -1 & -2 \\
        0 & 0 & 0 & 0 \\
        0 & 0 & 0 & 0
      \end{pmatrix} = U.
    \]
    So we have an upper triangular matrix $U$.

    Now write out the elementary matrices used to transform $A$ to $U$. For 
    example, I used
    \[
      E_1 =
      \begin{pmatrix}
        1 & 0 & 0 & 0 \\
        -2 & 1 & 0 & 0 \\
        0 & 0 & 1 & 0 \\
        0 & 0 & 0 & 1
      \end{pmatrix},
      \ E_2 =
      \begin{pmatrix}
        1 & 0 & 0 & 0 \\
        0 & 1 & 0 & 0 \\
        -3 & 0 & 1 & 0 \\
        0 & 0 & 0 & 1
      \end{pmatrix},
      \ E_3
      \begin{pmatrix}
        1 & 0 & 0 & 0 \\
        0 & 1 & 0 & 0 \\
        0 & 0 & 1 & 0 \\
        -4 & 0 & 0 & 1
      \end{pmatrix}
    \]
    in the first step and then
    \[
      E_4 =
      \begin{pmatrix}
        1 & 0 & 0 & 0 \\
        0 & 1 & 0 & 0 \\
        0 & -2 & 1 & 0 \\
        0 & 0 & 0 & 1
      \end{pmatrix},
      \ E_5 =
      \begin{pmatrix}
        1 & 0 & 0 & 0 \\
        0 & 1 & 0 & 0 \\
        0 & 0 & 1 & 0 \\
        0 & -3 & 0 & 1
      \end{pmatrix}.
    \]
    So $E_5 E_4 E_3 E_2 E_1 A = U$. Rearranging, we have $A = E_1^{-1} E_2^{-1} 
    E_3^{-1} E_4^{-1} E_5^{-1} U$. Then compute
    \[
      E_1^{-1} E_2^{-1} E_3^{-1} E_4^{-1} E_5^{-1} = \dots =
      \begin{pmatrix}
        1 & 0 & 0 & 0 \\
        2 & 1 & 0 & 0 \\
        3 & 2 & 1 & 0 \\
        4 & 3 & 0 & 1
      \end{pmatrix} = L.
    \]
    Now check that $LU = A$ to ensure that your solution is correct!

    \textbf{Note 1.} We get $L$ to be lower triangular because none of the 
    elementary row operations used were row-swaps. This is because the 
    elementary matrix for a row swap has a nonzero element above the diagonal.

    \textbf{Note 2.} Also note that we have deliberately chosen the elementary 
    row operations to work down from the top of columns, i.e. if we have a 
    corresponding elementary matrix with nonzero entry \emph{above} the 
    diagonal, then we wouldn't get a lower triangular matrix.

    \textbf{Note 3.} Writing $A = LU$ is called $LU$ factorisation---see Poole 
    section 3.4. In fact, there is a \emph{much} quicker method (but 
    equivalent) to finding $L$ on page 181 of Poole. I don't want to confuse 
    anyone, so I won't go into that method here. However, it is quite easy to 
    understand if you have a few minutes to spare.
  \item[B7] Let $V$ denote the vector space which is the span of the set $S = 
    \set{\upe^{\lambda x} | \lambda \in \reals}$. Prove that $V$ does not have 
    finite dimension. (It is not clear which field we are working in, but we 
    assume that $x$ takes values in $\reals$.)

    \textbf{Note.} To understand/believe the following proof, you need to be 
    comfortable with limits. I do not think this is unfair, since a limit 
    appears in the previous question. See sequences and series.

    \textbf{Proof.} Suppose for contraction that $V$ is finite dimensional, so 
    suppose that $B = \set{\upe^{\lambda_1 x}, \upe^{\lambda_2 x}, \dots, 
    \upe^{\lambda_n x}}$ is a basis for some $\lambda_1, \dots, \lambda_n \in 
    \reals$. We may, without loss of generality, suppose that $\lambda_1 < 
    \lambda_2 < \dots < \lambda_n$.
    
    We now want to show that there is some element in $V$ that is not in 
    $\spn(B)$. First, recall that $\upe^{\lambda x}$ grows exponentially fast 
    as $x$ increases/decays exponentially fast as $x$ decreases. Let's consider 
    any $\lambda_{n + 1} \in \reals$, where $\lambda_{n + 1} > \lambda_n$ (so 
    $\lambda_{n + 1} > \lambda_i$ for each $i = 1, \dots, n$). Suppose that 
    there exists $a_1, a_2, \dots, a_n \in \reals$ such that
    \[
      \upe^{\lambda_{n + 1} x} = a_1 \upe^{\lambda_1 x} + a_2 \upe^{\lambda_2 
      x} + \dots + a_n \upe^{\lambda_n x}.
    \]
    Dividing through by $\upe^{\lambda_{n + 1} x}$, we get
    \begin{equation} \label{eqn:sum2}
      1 = a_1 \upe^{(\lambda_1 - \lambda_{n + 1}) x} + a_2 \upe^{(\lambda_2 - 
        \lambda_{n + 1}) x} + \dots + a_n \upe^{(\lambda_n - \lambda_{n + 1}) 
      x}.
    \end{equation}
    (When we are dealing with functions, any identities like this must hold for 
    all $x \in \reals$.)
    
    We observe that $\lambda_i - \lambda_{n + 1} < 1$ for each $i = 1, 2, 
    \dots, n$. So terms of the form $\upe^{(\lambda_i - \lambda_{n + 1}) x}$ 
    tend to zero as $x \to \infty$. Therefore, taking the limit 
    of~\eqref{eqn:sum2} as $x \to \infty$, we have
    \[
      1 = \lim_{x \to \infty} a_1 \upe^{(\lambda_1 - \lambda_{n + 1}) x} + a_2 
      \upe^{(\lambda_2 - \lambda_{n + 1}) x} + \dots + a_n \upe^{(\lambda_n - 
        \lambda_{n + 1}) x} = 0,
    \]
    a contradiction.
    
    Therefore there does not exist a finite basis for $V$ and so we deduce that 
    $V$ is infinite dimensional. \hfill $\square$

    \textbf{Note 1.} This proof requires a bit of ingenuity (!), which is why a 
    relatively short proof is probably worth 20 marks. Since there are no 
    `optional' questions on this year's exam, I can't see such a question 
    appearing this year. But as usual, don't take my word for it.

    \textbf{Note 2.} When you see exponential terms appearing in questions, you 
    should think about:
    \begin{enumerate}
      \item Can I use logarithms? (Not in this question, because we have the 
        sum of $\upe^{ax}$ terms, which doesn't give us anything easier to work 
        with.)
      \item If $x = 0$, then $\upe^{\lambda x} = 1$.
      \item How fast do things grow or decay?
      \item Possibly trigonometric functions.
    \end{enumerate}
    In fact, the main part of this proof involves showing that any sum of 
    exponential functions will not give you an exponential with a different 
    term in the exponent.
\end{itemize}

\end{document}

