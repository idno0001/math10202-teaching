\documentclass[english,12pt,a4paper]{scrartcl}
\usepackage{multicol}
\usepackage{mdframed}
\usepackage[cm]{anton}

\renewcommand{\vec}[1]{\bm{#1}}

\title{MATH10202 linear algebra}
\subtitle{2012 exam}
\author{}
\date{\vspace{-5ex}}

\DeclareMathOperator{\spn}{span}
\newcommand\spanset[1]{\ensuremath{\spn\set{#1}}}

\begin{document}
\maketitle

\begin{itemize}
  \item[A3 (i)] Let $A$ be the following $4 \times 4$ matrix:
    \[
      A =
      \begin{pmatrix}
        1 & 2 & 2 & 3 \\
        2 & 3 & 3 & 4 \\
        3 & 4 & 4 & 5 \\
        4 & 5 & 5 & 6
      \end{pmatrix}.
    \]
    Find a lower triangular matrix $L$ and an upper triangular matrix $U$ so 
    that $A = LU$.

    \textbf{Rough solution.} We want to write $A$ as the product of a lower and 
    upper triangular matrix. Thus, your first instinct should be to row reduce 
    $A$ to an upper triangular matrix. Don't worry about getting it to 
    (reduced) row-echelon form---here, we don't need leading $1$s in the rows.  

    Begin with
    \[
      \begin{pmatrix}
        1 & 2 & 2 & 3 \\
        2 & 3 & 3 & 4 \\
        3 & 4 & 4 & 5 \\
        4 & 5 & 5 & 6
      \end{pmatrix}
      \refrel{$\substack{r_2 \to r_2 - 2r_1 \\ r_3 \to r_3 - 3r_1 \\ r_4 \to 
      r_4 - 4r_1}$}{\to}
      \begin{pmatrix}
        1 & 2 & 2 & 3 \\
        0 & -1 & -1 & -2 \\
        0 & -2 & -2 & -4 \\
        0 & -3 & -3 & -6
      \end{pmatrix}
      \refrel{$\substack{r_3 \to r_3 - 2r_2 \\ r_4 \to r_4 -3r_2}$}{\to}
      \begin{pmatrix}
        1 & 2 & 2 & 3 \\
        0 & -1 & -1 & -2 \\
        0 & 0 & 0 & 0 \\
        0 & 0 & 0 & 0
      \end{pmatrix} = U.
    \]
    So we have an upper triangular matrix $U$.

    Now write out the elementary matrices used to transform $A$ to $U$. For 
    example, I used
    \[
      E_1 =
      \begin{pmatrix}
        1 & 0 & 0 & 0 \\
        -2 & 1 & 0 & 0 \\
        0 & 0 & 1 & 0 \\
        0 & 0 & 0 & 1
      \end{pmatrix},
      \ E_2 =
      \begin{pmatrix}
        1 & 0 & 0 & 0 \\
        0 & 1 & 0 & 0 \\
        -3 & 0 & 1 & 0 \\
        0 & 0 & 0 & 1
      \end{pmatrix},
      \ E_3
      \begin{pmatrix}
        1 & 0 & 0 & 0 \\
        0 & 1 & 0 & 0 \\
        0 & 0 & 1 & 0 \\
        -4 & 0 & 0 & 1
      \end{pmatrix}
    \]
    in the first step and then
    \[
      E_4 =
      \begin{pmatrix}
        1 & 0 & 0 & 0 \\
        0 & 1 & 0 & 0 \\
        0 & -2 & 1 & 0 \\
        0 & 0 & 0 & 1
      \end{pmatrix},
      \ E_5 =
      \begin{pmatrix}
        1 & 0 & 0 & 0 \\
        0 & 1 & 0 & 0 \\
        0 & 0 & 1 & 0 \\
        0 & -3 & 0 & 1
      \end{pmatrix}.
    \]
    So $E_5 E_4 E_3 E_2 E_1 A = U$. Rearranging, we have $A = E_1^{-1} E_2^{-1} 
    E_3^{-1} E_4^{-1} E_5^{-1} U$. Then compute
    \[
      E_1^{-1} E_2^{-1} E_3^{-1} E_4^{-1} E_5^{-1} = \dots =
      \begin{pmatrix}
        1 & 0 & 0 & 0 \\
        2 & 1 & 0 & 0 \\
        3 & 2 & 1 & 0 \\
        4 & 3 & 0 & 1
      \end{pmatrix} = L.
    \]
    Now check that $LU = A$ to ensure that your solution is correct!

    \textbf{Note 1.} We get $L$ to be lower triangular because none of the 
    elementary row operations used were row-swaps. This is because the 
    elementary matrix for a row swap has a nonzero element above the diagonal.

    \textbf{Note 2.} Also note that we have deliberately chosen the elementary 
    row operations to work down from the top of columns, i.e. if we have a 
    corresponding elementary matrix with nonzero entry \emph{above} the 
    diagonal, then we wouldn't get a lower triangular matrix.

    \textbf{Note 3.} Writing $A = LU$ is called $LU$ factorisation---see Poole 
    section 3.4. In fact, there is a \emph{much} quicker method (but 
    equivalent) to finding $L$ on page 181 of Poole. I don't want to confuse 
    anyone, so I won't go into that method here. However, it is quite easy to 
    understand if you have a few minutes to spare.
  \item[B7] Let $V$ denote the vector space which is the span of the set $S = 
    \set{\upe^{\lambda x} | \lambda \in \reals}$. Prove that $V$ does not have 
    finite dimension.

    \textbf{Note.} To understand/believe the following proof, you need to be 
    comfortable with limits. See sequences and series.

    \textbf{Proof.} First, note that if $S$ is linearly independent, we can 
    deduce that it is a basis for $V$ (since $\spn(S) = V$). So we show that 
    $S$ is linearly independent.

    Suppose that for distinct $\lambda_1 < \lambda_2 < \dots < \lambda_n \in 
    \reals$ and some $a_1, a_2, \dots, a_n \in \reals$ we have
    \begin{equation} \label{eqn:sum1}
      a_1 \upe^{\lambda_1 x} + \dots + a_n \upe^{\lambda_n x} \equiv 0.
    \end{equation}

    Dividing by the largest term $\upe^{\lambda_n x}$ we get
    \[
      a_1 \upe^{(\lambda_1 - \lambda_n) x} + \dots + a_{n - 1} 
      \upe^{(\lambda_{n - 1} - \lambda_n) x} + a_n \equiv 0,
    \]
    and this holds for all $x \in \reals$. So for any $i \in \set{1, 2, \dots, 
    n - 1}$ we have $\upe^{(\lambda_i - \lambda_n) x} \to 0$ as $x \to \infty$.  
    (Convince yourself of this.) So when $x$ is very large we are left with
    \begin{equation} \label{eqn:limsum}
      0 = \lim_{x \to \infty} a_1 \upe^{(\lambda_1 - \lambda_n) x} + \dots + 
      a_{n - 1} \upe^{(\lambda_{n - 1} - \lambda_n) x} + a_n = a_n.
    \end{equation}

    Repeating this argument for the remaining terms, we get that
    \[
      a_1 = a_2 = \dots = a_{n - 1} = a_n = 0,
    \]
    Therefore $S$ is linearly independent and is hence a basis for $V$. Since 
    $S$ consists of infinitely many terms, it follows that $V$ is infinite 
    dimensional. \hfill $\square$

    \textbf{Note 1.} This proof requires a bit of ingenuity (!), which is why a 
    relatively short proof is probably worth 20 marks. Since there are no 
    `optional' questions on this year's exam, I can't see such a question 
    appearing this year. But as usual, don't take my word for it.

    \textbf{Note 2.} When you see exponential terms appearing in questions, you 
    should think about:
    \begin{enumerate}
      \item Can I use logarithms? (Not in this question, because we have the 
        sum of $\upe^{ax}$ terms, which doesn't give us anything easier to work 
        with.)
      \item What happens if I set $x = 0$? (If you do this in~\eqref{eqn:sum1}, 
        you learn something about the coefficients.)
      \item How fast do things grow?
    \end{enumerate}
    In this question, the last point is the most important. In 
    fact,~\eqref{eqn:limsum} means: ``The largest term eventually dominates the 
    other terms.'' This means that, if you drew the graph of $a_1 
    \upe^{\lambda_1 x} + \dots + a_n \upe^{\lambda_n x}$ (going along in the 
    positive $x$-direction), it would eventually look like $a_n \upe^{\lambda_n 
    x}$. This is why exponential growth (or decay) is so interesting in 
    mathematics. See sequences and series!
\end{itemize}

\end{document}

