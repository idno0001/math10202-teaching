\documentclass[english,12pt,a4paper]{scrartcl}
\usepackage{multicol}
\usepackage[cm]{anton}

\renewcommand{\vec}[1]{\bm{#1}}

\title{MATH10202 Linear Algebra A}
\subtitle{Supplementary exercise sheet 4: determinants and eigenvalues}
\author{}
\date{\vspace{-4.5ex}}

% Arabic numbering, then roman. (Underlying stuff: the enumitem package.)
\setlist[enumerate,1]{label=S4.\arabic*., ref=S4.\arabic*}
\setlist[enumerate,2]{label=(\roman*), ref=(\roman*)}

% Starred questions: use \begin{modenumerate} .. .. \end{modenumerate} and then 
% add stars to questions using \moditem{*}.
% Source: 
% http://tex.stackexchange.com/questions/21004/add-asterisk-after-labels-in-enumerate/21010#21010
\newenvironment{modenumerate}
  {\enumerate\setupmodenumerate}
  {\endenumerate}
\newif\ifmoditem
\newcommand{\setupmodenumerate}{%
  \global\moditemfalse
  \let\origmakelabel\makelabel
  \def\moditem##1{\global\moditemtrue\def\mesymbol{##1}\item}% 
  \def\makelabel##1{%
  \origmakelabel{##1\ifmoditem\rlap{\mesymbol}\fi\enspace}% 
\global\moditemfalse}%
}

\begin{document}
\maketitle

\vspace{-5ex}
\noindent \textbf{Disclaimer:} The following exercises may contain material 
outside the scope of the course.

\section*{Determinants}
\begin{enumerate}
  \item % Q7.5
    In the following, use the properties of determinants to evaluate the given 
    determinant by inspection, noticing certain special properties.  Explain 
    your reasoning.
    \begin{multicols}{4}
      \begin{enumerate}
        \item $
          \begin{vmatrix}
            3 & 1 & 0 \\
            0 & -2 & 5 \\
            0 & 0 & 4
          \end{vmatrix}
          $
        \item $
          \begin{vmatrix}
            2 & 3 & -4 \\
            1 & -3 & -2 \\
            -1 & 5 & 2
          \end{vmatrix}
          $
        \item $
          \begin{vmatrix}
            4 & 1 & 3 \\
            -2 & 0 & -2 \\
            5 & 4 & 1
          \end{vmatrix}
          $
        \item $
          \begin{vmatrix}
            0 & 2 & 0 & 0 \\
            -3 & 0 & 0 & 0 \\
            0 & 0 & 0 & 4 \\
            0 & 0 & 1 & 0
          \end{vmatrix}
          $.
      \end{enumerate}
    \end{multicols}
  \item % Q7.6
    Suppose that
    \[
      \begin{vmatrix}
        a & b & c \\
        d & e & f \\
        g & h & i
      \end{vmatrix}
       = 4.
    \]
    Find the following determinants.
    \begin{multicols}{3}
      \begin{enumerate}
        \item $
          \begin{vmatrix}
            2a & 2b & 2c \\
            d & e & f \\
            g & h & i
          \end{vmatrix}
          $
        \item $
          \begin{vmatrix}
            d & e & f \\
            a & b & c \\
            g & h & i
          \end{vmatrix}
          $
        \item $
          \begin{vmatrix}
            2c & b & a \\
            2f & e & d \\
            2i & h & g
          \end{vmatrix}
          $.
      \end{enumerate}
    \end{multicols}
  \item % Q7.7
    Find all values of $k$ for which
    \[
      A =
      \begin{pmatrix}
        k & -k & 3 \\
        0 & k + 1 & 1 \\
        k & -8 & k - 1
      \end{pmatrix}
    \]
    is invertible.
  \item % Q7.8
    Suppose that $A$ and $B$ are $n \times n$ matrices, with $\det(A) = 3$ and 
    $\det(B) = -2$. Find the following determinants.
    \begin{multicols}{3}
      \begin{enumerate}
        \item $\det(AB)$
        \item $\det(B^{-1}A)$
        \item $\det(3B^T)$.
      \end{enumerate}
    \end{multicols}
  \item % Q7.10
    Suppose that $A$ and $B$ are $n \times n$ matrices.
    \begin{enumerate}
      \item Prove that $\det(AB) = \det(BA)$.
      \item If $A$ is \key{idempotent} (that is, $A^2 = A$), find all possible 
        values of $\det(A)$.
    \end{enumerate}
\end{enumerate}

\section*{Eigenvalues, eigenvectors and eigenspaces}
\begin{enumerate}[start=6]
  \item % Q8.2
    Suppose that $A$ is a $2 \times 2$ matrix with eigenvectors
    \[
      \vec{v_1} =
      \begin{pmatrix}
        1 \\ -1
      \end{pmatrix}
      \quad \text{and} \quad
      \vec{v_2} =
      \begin{pmatrix}
        1 \\ 1
      \end{pmatrix}
    \]
    corresponding to eigenvalues $\lambda_1 = \frac{1}{2}$ and $\lambda_2 = 2$, 
    respectively. Let
    \[
      \vec{x} =
      \begin{pmatrix}
        5 \\ 1
      \end{pmatrix}.
    \]
    Find $A^{10}\vec{x}$.
  \item % Q8.3
    Suppose that $A$ is a $3 \times 3$ matrix with eigenvectors
    \[
      \vec{v_1} =
      \begin{pmatrix}
        1 \\ 0 \\ 0
      \end{pmatrix}
      ,
      \quad
      \vec{v_2} =
      \begin{pmatrix}
        1 \\ 1 \\ 0
      \end{pmatrix}
      ,
      \quad
      \vec{v_3} =
      \begin{pmatrix}
        1 \\ 1 \\ 1
      \end{pmatrix}
    \]
    corresponding to eigenvalues $\lambda_1 = -\frac{1}{3}$, $\lambda_2 = 
    \frac{1}{3}$ and $\lambda_3 = 1$, respectively. Let
    \[
      \vec{x} =
      \begin{pmatrix}
        2 \\ 1 \\ 2
      \end{pmatrix}.
    \]
    Find $A^{20}\vec{x}$.
  \item % Q8.4
    \begin{enumerate}
      \item Let $A$ be a square matrix $A$. Show that $A$ and $A^T$ have the 
        same characteristic polynomial and hence the same eigenvalues.
      \item Give an example of a $2 \times 2$ matrix $A$ for which $A^T$ and 
        $A$ have different eigenspaces.
    \end{enumerate}
  \item % Q8.5
    Let $A$ be an idempotent matrix (that is, $A^2 = A$). Show that $\lambda = 
    0$ and $\lambda = 1$ are the only possible eigenvalues of $A$. [Hint: 
    consider eigenvectors.]
  \item \label{q8.6} % Q8.6
    Let $\vec{v}$ be an eigenvector of a matrix $A$ corresponding to eigenvalue 
    $\lambda$. Prove that $\vec{v}$ is also an eigenvector for $A + kI$ 
    corresponding to eigenvalue $\lambda + k$.
  \item % Q8.7
    Let
    \[
      A =
      \begin{pmatrix}
        3 & 2 \\
        5 & 0
      \end{pmatrix}.
    \]
    \begin{enumerate}
      \item Find the eigenvalues and eigenspaces of $A$.
      \item Using \ref{q8.6} and any other results that you may know, find the 
        eigenvalues and eigenspaces of $A^{-1}$, $A - 2I$ and $A + 2I$.
    \end{enumerate}
  \item % Q8.8
    If $A$ and $B$ are row equivalent matrices, do they necessarily have the 
    same eigenvalues? If yes, prove it; otherwise, give a counterexample.
\end{enumerate}

\end{document}

